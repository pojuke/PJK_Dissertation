\chapter{Introduction}
%\renewcommand{\sectionmark}[1]{}
%\chaptermark{Introduction}
\fancyhead[LE, RO]{\thepage}
\fancyhead[RE]{CHAPTER 1}
\fancyhead[LO]{INTRODUCTION}
\fancyfoot{}
\renewcommand{\headrulewidth}{0pt}
\setlength{\parindent}{1cm}


\section{General introduction}
What causes species diversity to vary among different places? Answers to this question can help sustain biodiversity, which has been a significant challenge under the aggravating threat of species loss. In response, ecologists have devised a long list of theories to explain the maintenance of species diversity \citep{Vellend2016}. Community ecologists have expanded from the classic focus of species competition (e.g., \citealp{Gause1934}) to consider the effects of non-competitive interactions and multiple trophic levels (e.g., \citealp{Chesson2008b, Mccann2011, Bascompte2013}), thus better reflect the complexity of natural systems. However, as multiple mechanisms operate simultaneously to shape species diversity \citep{Amarasekare2007}, developing a general predictive framework will be challenging if theories continue to advance with little crosstalk.
\par


Plants and their soil microbial communities represent one association where the players are engaged in complex interactions with consequences that are not fully explored. Plants interact with a wide variety of soil microbes, ranging from pathogenic to mutualistic microbes with various degrees of host specificity \citep{Bever2010, vanderPutten2013}. Plants can also indirectly influence the performance of nearby competitors by modifying soil microbial communities, a phenomenon commonly studied under the framework of plant--soil feedback \citep{Bever1997, Bever2003}. Furthermore, these plant--soil microbe interactions are inherently age-dependent as it takes time for plants to condition the soil \citep{Kardol2013}. Explicit consideration of the temporal dynamics of these interactions can have critical implications for both natural and agricultural systems, guiding plant restoration and agricultural practices \citep{Kulmatiski2006, Mariotte2018}. However, this aspect of plant--soil microbe interactions remains rarely studied. 
\par 


In this dissertation, I explore the complementarity of different frameworks for understanding species coexistence and, with plant--soil microbe associations as an example, highlight the multi-trophic and age-dependent nature of ecological communities. 
\par



\section{Overview}
I start by exploring the theoretical and empirical relationships among contemporary niche theory and modern coexistence theory, which are two powerful frameworks for understanding the niche's role in species coexistence \citep{Chesson2000, Chase2003}. Contemporary niche theory provides a mechanistic understanding of how the resource supply ratio and trade-offs in species' impact niche and requirement niches affect competitive outcomes; modern coexistence theory formularizes coexistence as the balance between equalizing (i.e., lower fitness differences) and stabilizing (i.e., greater niche differences) processes. In Chapter 2, I ask how the criteria for coexistence under contemporary niche theory translate into the stabilizing and equalizing processes of modern coexistence theory. I show that varying resource supply ratios reflect an equalizing process, varying impact niche overlap reflects a stabilizing process, and varying requirement niche overlap may be both stabilizing and equalizing, but has no qualitative effect on coexistence.
In Chapter 3, I extend the approach developed in Chapter 2 to discuss how priority effects, the phenomenon where species arrival order affects competitive outcomes \citep{Fukami2015}, fit within the stabilizing and equalizing concepts of modern coexistence theory. I argue that the only compatible priority effects are those characterized by positive frequency dependence, irrespective of whether they emerge in equilibrium or non-equilibrium systems.
By exploring the connections among different frameworks of species coexistence, the two chapters together lay a foundation for further conceptual advances to incorporate more complicated interactions.
\par


In Chapter 4, I ask how the interactions between plants and soil microbes influence plant coexistence. It is known that systems vary in the impacts microbes have on plants and in the ways plants compete with each other, rendering a general predictive theory difficult \citep{Lekberg2018}. Building on the previous two chapters, I argue that the concepts of niche and fitness difference from modern coexistence theory should be used to contextualize how soil microbes contribute to plant coexistence \citep{KeMiki2015}. 
With a general plant--soil microbe interaction model, I show that, depending on host-specificity, both pathogens and mutualists can affect the niche difference between competing plants. Moreover, soil microbes can affect plant fitness differences and modify the importance of plant--plant competition for determining plant coexistence, a role that is often overlooked by the literature. I then propose experimental designs that can efficiently measure both plant--soil microbe and plant--plant interactions. With an empirical case study, I demonstrate how the proposed predictive framework can provide a better way to identify the actual processes through which soil microbes affect plant coexistence.
\par


In Chapter 5 and 6, I focus on the temporal development of plant--soil microbe interactions at Bodega Bay, a coastal foredune community dominated by four perennial species: the introduced grass \textit{Ammophila arenaria} (Poaceae), the introduced succulent dwarf-shrub \textit{Carpobrotus edulis} (Aizoaceae), and the native shrubs \textit{Baccharis pilularis} (Asteraceae) and \textit{Lupinus arboreus} (Fabaceae). I use a series of high-resolution aerial photos, which were taken annually from 1992 to 2016, to estimate the age of individual plants \citep{Danin1998}. From the age estimation, I reconstruct a chronosequence of soil conditioning length, ranging between 1 to 11 years for \textit{L. arboreus} and 2 to 25 years for the other three species.
In Chapter 5, I examine the successional dynamics of soil fungal communities associated with \textit{C. edulis} and \textit{L. arboreus}. By collecting soil from plant individuals along the chronosequence and re-sampling a subset of individuals for three consecutive years, I test the hypothesis that a deterministic force is driving the convergence of soil fungal communities \citep{Connell1977, DiniAndreote2015, Li2016}. 
I show that the beta diversity among fungal communities decreased and we are able to predict the fungal community composition more accurately as plants aged. I argue that the combination of chronosequence and longitudinal data strengthened the inference of an underlying deterministic process shaping soil fungal communities.
\par 


Despite our growing knowledge on the successional dynamics of soil microbes, how their effects on plants vary with conditioning time remain rarely studied \citep{Kardol2013, Lepinay2018}. This is presumably because preparing soils with different conditioning lengths is often not feasible in the greenhouse \citep{Kardol2013, Kulmatiski2018}, and the length of soil conditioning in the field can only be quantified with coarse resolution (e.g., \citealp{Day2015, Speek2015}). 
Taking advantage of the chronosequence, in Chapter 6, I study the turnover of soil microbial communities associated with all four dominant plants at Bodega Bay. I then design a greenhouse experiment that preserved age-specific microbial communities to study how their turnover affected plant performance.
I show that compositional turnovers of soil microbes translated into temporally varying plant--soil microbe interactions, and different plant--soil combinations showed different temporal development patterns. 
Using a general individual-based model \citep{Fukami2013}, I further show that the temporal development patterns of microbial effects alter the transient dynamics of plant community assembly. 
With the combination of high-throughput sequencing, greenhouse experiment, and simulation models, Chapter 5 and 6 together develop a dynamic perspective of plant--soil feedback to understand their long-term consequences on plant community dynamics.
\par


\section{Overall conclusion and future perspectives}
Two common themes emerge from this dissertation. First, mechanisms that maintain species diversity often operate simultaneously and are tightly coupled \citep{Amarasekare2007, Letten2018}. For example, plants are engaged in plant--plant and plant--soil microbe interactions at the same time, and the strength of one process affects the relative importance of the other (Chapter 4). To study how multiple mechanisms interactively determine species coexistence, I contextualize the effects of different mechanisms with higher-level processes proposed in general coexistence theories (see also \citealp{Vellend2016}). In particular, I show how differences in resource consumption traits (Chapter 2 and 3) and microbial traits (Chapter 4) translate to niche and fitness differences of modern coexistence theory, and how the two components can be used as a common currency for understanding coexistence. As the list of theories explaining the maintenance of species diversity continues to increase, this approach provides a route to synthesize the overall effects of different trophic levels and interaction types \citep{Bartomeus2018, Lanuza2018}.
\par


Another common theme among the chapters is that the strength of species interaction is not constant: it varies depending on the arrival order of species (Chapter 3 and 4; \citealp{Fukami2015, Duhamel2019}), the timing of the interaction, and the age-/stage-structure within species' populations (Chapter 5 and 6; \citealp{Kardol2013Oikos, Peay2018}). Using plant--soil microbe interactions as an example, I show that the microbial community structure (Chapter 5) and their effects on plant performance (Chapter 6) vary with the duration of soil conditioning, and the temporal development patterns of these interactions have a strong effect on community dynamics. The fact that interaction strengths are time- and age-dependent is not unique to plant--soil microbe interactions but a general phenomenon that applies to almost all multi-cellular organisms \citep{Miller2011, deRoos2013, Nakazawa2015}. In some cases, not only the strength but also the type of species interactions change as individuals mature \citep{Yang2010, KeNakazawa2018}. Overall, my results highlight the temporal complexity of ecological communities and call for more understanding on how species' interactions are coordinated in time.
\par


My future work continues to explore the temporal dimensions of species interactions. Moving forward, I aim to study how ecological processes with different dynamical rates interactively affect community stability. When studying the effects of multi-trophic interactions on species coexistence, theories often assume that non-focal trophic levels change with rates much faster than the focal community (e.g., \citealp{Chesson2008}; see also Chapter 4). However, species turnover and ecological processes can vary significantly in their dynamical rates \citep{Rinaldi2000, Menge2012, LiChesson2016}. Consider the belowground interactions among plants as an example, soil microbial succession can occur at various timescales. Moreover, while plant--plant competition for nutrient uptake may be a fast process, nutrient release from plant litter can be extremely slow \citep{Menge2008}. By combining empirical and theoretical approaches, as done in this dissertation, I hope to open new avenues to develop a temporally-explicit perspective of community ecology.
\par



\section{Author contributions}
Throughout my dissertation, I worked closely with many collaborators whose contributions are detailed below. These collaborators are listed as co-authors on the peer-reviewed manuscripts (or manuscripts in preparation) corresponding to each of my dissertation chapters.
\par


\noindent \textbf{Chapter 2 -- Linking modern coexistence theory and contemporary niche theory}\\
\noindent Andrew Letten and Tadashi Fukami worked with me to develop the ideas for the research. I designed the simulation and performed the mathematical analysis. All authors contributed to the writing and revision of the manuscript. Funding supporting this work came from the Center for Computational, Evolutionary, and Human Genomics (CEHG), the Department of Biology and the Terman Fellowship of Stanford University, and the National Science Foundation (NSF). The research has been published as a peer-reviewed manuscript in \textit{Ecological Monographs}, co-first authored by Andrew Letten and me.
\bigskip


\noindent \textbf{Chapter 3 -- Coexistence theory and the frequency-dependence of priority effects}\\
\noindent Andrew Letten worked with me to design the research, assisted in numerical analyses, and contributed to manuscript writing and revision. Funding supporting this work came from the Center for Computational, Evolutionary, and Human Genomics (CEHG), the Department of Biology of Stanford University, and the Studying Abroad Scholarship from the Ministry of Education, Taiwan. The research has been published as a peer-reviewed manuscript in \textit{Nature Ecology \& Evolution}.
\bigskip


\noindent \textbf{Chapter 4 -- Effects of soil microbes on plant competition: a perspective from modern coexistence theory}\\
\noindent Joe Wan worked with me to develop the research and formulate the analysis. He also worked with me on manuscript preparation and revision. Funding supporting this work came from the Department of Biology of Stanford University and the Studying Abroad Scholarship from the Ministry of Education, Taiwan. 
\bigskip


\noindent \textbf{Chapter 5 -- Testing chronosequence predictions with longitudinal data reveals microbial community convergence}\\
\noindent Tadashi Fukami worked with me to design the research. Tadashi Fukami and J. Nick Hendershot assisted in data analysis and manuscript preparation. Funding supporting this work came from the Department of Biology and the Terman Fellowship of Stanford University, the National Science Foundation (NSF), and the Studying Abroad Scholarship from the Ministry of Education, Taiwan.
\bigskip


\noindent \textbf{Chapter 6 -- Dynamic plant--soil microbe interactions: the neglected effect of soil conditioning time}\\
\noindent Tadashi Fukami worked with me to design the research and assisted in analyzing the empirical data. Peter Zee helped me develop the individual-based model and performed numerical simulations. Tadashi Fukami and Peter Zee contributed to manuscript preparation and revision. Funding supporting this work came from the Department of Biology and the Terman Fellowship of Stanford University, and the National Science Foundation (NSF), and the Studying Abroad Scholarship from the Ministry of Education, Taiwan.



%%%%%%%%%%%%%%%%
% Text deleted and not useful %%
%%%%%%%%%%%%%%%%

%% ABSTRACTS
%% Chapter 1
% Modern coexistence theory and contemporary niche theory represent parallel frameworks for understanding the niche's role in species coexistence. Despite increasing prominence and shared goals, their compatibility and complementarity have received little attention. This paucity of overlap not only presents an obstacle to newcomers to the field, but it also precludes further conceptual advances at their interface. Here, we present a synthetic treatment of the two frameworks. We review their main concepts and explore their theoretical and empirical relationship, focusing on how the resource supply ratio, impact niche, and requirement niche of contemporary niche theory translate into the stabilizing and equalizing processes of modern coexistence theory. We show for a general consumer--resource model that varying resource supply ratios reflects an equalizing process; varying impact niche overlap reflects a stabilizing process; and varying requirement niche overlap may be both stabilizing and equalizing, but has no qualitative effect on coexistence. These generalizations provide mechanistic insight into modern coexistence theory, while also clarifying the role of contemporary niche theory's impacts and requirements in mediating coexistence. From an empirical perspective, we recommend a hierarchical approach, in which quantification of the strength of stabilizing mechanisms is used to guide more focused investigation into the underlying niche factors determining species coexistence. Future research that considers alternative assumptions, including different forms of species interaction, spatio-temporal heterogeneity, and priority effects, would facilitate a more complete synthesis of the two frameworks.
%%
%% Chapter 2
%Priority effects are commonly invoked to describe a broad suite of phenomena capturing the influence of species arrival order on the diversity, composition and function of ecological communities. Several studies have suggested reframing priority effects around the stabilizing and equalizing concepts of coexistence theory. We show that the only compatible priority effects are those characterized by positive frequency dependence, irrespective of whether they emerge in equilibrium or non-equilibrium systems. 
%%  
%% Chapter 3
% Growing evidence shows that soil microbes affect plant coexistence in a variety of systems. However, since these systems vary in the impacts microbes have on plants and in the ways plants compete with each other, it is challenging to integrate results into a general predictive theory. To this end, we suggest that the concepts of niche and fitness difference from modern coexistence theory should be used to contextualize how soil microbes contribute to plant coexistence. Synthesizing a range of mechanisms under a general plant--soil microbe interaction model, we show that, depending on host-specificity, both pathogens and mutualists can affect the niche difference between competing plants. However, we emphasize the need to also consider the effect of soil microbes on plant fitness differences, a role often overlooked when examining their role in plant coexistence. Additionally, since our framework predicts that soil microbes modify the importance of plant--plant competition relative to other factors for determining the outcome of competition, we suggest that experimental work should simultaneously quantify microbial effects and plant competition. Thus, we propose experimental designs that efficiently measure both processes and show how our framework can be applied to identify the underlying drivers of coexistence. Using an empirical case study, we demonstrate that the processes driving coexistence can be counterintuitive, and that our general predictive framework provides a better way to identify the true processes through which soil microbes affect coexistence.
%%  
%% Chapter 4
%%  
%% Chapter 5
% Plant--soil feedbacks (PSF), the reciprocal interactions between plants and soil microbes, shape the structure of plant communities, but how the length of soil conditioning affects PSF strength is rarely studied. Using a chronosequence reconstructed from aerial photos, we characterized the soil microbial communities associated with four perennial dune plants and their turnover as plants aged. We also studied the resulting PSF in a greenhouse experiment that preserved age-specific soil properties. For all plants, we found that their microbial communities changed with increasing conditioning time. These compositional turnovers translated into a temporally varying PSF, and different plant--soil combinations showed different temporal development patterns. With an general individual-based model, we further found that the temporal development patterns of PSF affected the convergence rate of plant community assembly. Our results suggest that future studies should take a dynamic perspective to understand the long-term consequences of PSF on plant community dynamics.
