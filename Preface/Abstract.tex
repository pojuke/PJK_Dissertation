\prefacesection{Abstract}
What causes species diversity to vary among different places? Answers for this question can help sustain biodiversity and its stability, which has been a significant challenge under aggravating threat of species loss. In response, ecologists have devised a long list of theories to explain the maintenance of species diversity. Community ecologists have expanded from the classic focus of species competition to consider the effects of non-competitive interactions and multiple trophic levels, thus better reflecting the complexity of natural systems. However, as multiple mechanisms operate simultaneously to shape species diversity, a general predictive framework would be difficult if theories continue to be developed with little crosstalk. Plants and their soil microbial communities represent one association where the players are engaged in complex interactions that remain not fully explored. Plants interact with a wide variety of soil microbes, ranging from pathogenic to mutualistic microbes that have various degrees of host specificity. Plants can also indirectly influence the performance of nearby competitors by modifying soil microbial communities, which is commonly studied under the framework of plant--soil feedback. Moreover, these plant--soil microbe interactions are inherently age-dependent as it takes time for plants to condition the soil, an aspect that remains rarely studied. Explicit consideration of the temporal dynamics of these interactions, however, can have critical implications in both natural and agricultural systems, guiding plant restoration, invasion control, and agricultural practices. In this dissertation, I explore the complementarity of different frameworks for understanding species coexistence and, with plant--soil microbe association as an example, highlight the multi-trophic and age-dependent nature of ecological communities. 

