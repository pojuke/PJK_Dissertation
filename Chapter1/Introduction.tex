\chapter{Introduction}
%\renewcommand{\sectionmark}[1]{}
%\chaptermark{Introduction}
\fancyhead[LE, RO]{\thepage}
\fancyhead[RE]{CHAPTER 1}
\fancyhead[LO]{INTRODUCTION}
\fancyfoot{}
\renewcommand{\headrulewidth}{0pt}



\section{Species coexistence}
Modern coexistence theory and contemporary niche theory represent parallel frameworks for understanding the niche's role in species coexistence. Despite increasing prominence and shared goals, their compatibility and complementarity have received little attention. This paucity of overlap not only presents an obstacle to newcomers to the field, but it also precludes further conceptual advances at their interface. Here, we present a synthetic treatment of the two frameworks. We review their main concepts and explore their theoretical and empirical relationship, focusing on how the resource supply ratio, impact niche, and requirement niche of contemporary niche theory translate into the stabilizing and equalizing processes of modern coexistence theory. We show for a general consumer-resource model that varying resource supply ratios reflects an equalizing process; varying impact niche overlap reflects a stabilizing process; and varying requirement niche overlap may be both stabilizing and equalizing, but has no qualitative effect on coexistence. These generalizations provide mechanistic insight into modern coexistence theory, while also clarifying the role of contemporary niche theory's impacts and requirements in mediating coexistence. From an empirical perspective, we recommend a hierarchical approach, in which quantification of the strength of stabilizing mechanisms is used to guide more focused investigation into the underlying niche factors determining species coexistence. Future research that considers alternative assumptions, including different forms of species interaction, spatio-temporal heterogeneity, and priority effects, would facilitate a more complete synthesis of the two frameworks.
\par


Priority effects are commonly invoked to describe a broad suite of phenomena capturing the influence of species arrival order on the diversity, composition and function of ecological communities. Several studies have suggested reframing priority effects around the stabilizing and equalizing concepts of coexistence theory. We show that the only compatible priority effects are those characterized by positive frequency dependence, irrespective of whether they emerge in equilibrium or non-equilibrium systems. 
\par



\section{Plant--soil microbe interactions}
Growing evidence shows that soil microbes affect plant coexistence in a variety of systems. However, since these systems vary in the impacts microbes have on plants and in the ways plants compete with each other, it is challenging to integrate results into a general predictive theory.
To this end, we suggest that the concepts of niche and fitness difference from modern coexistence theory should be used to contextualize how soil microbes contribute to plant coexistence. Synthesizing a range of mechanisms under a general plant--soil microbe interaction model, we show that, depending on host-specificity, both pathogens and mutualists can affect the niche difference between competing plants.
However, we emphasize the need to also consider the effect of soil microbes on plant fitness differences, a role often overlooked when examining their role in plant coexistence.
Additionally, since our framework predicts that soil microbes modify the importance of plant--plant competition relative to other factors for determining the outcome of competition, we suggest that experimental work should simultaneously quantify microbial effects and plant competition. Thus, we propose experimental designs that efficiently measure both processes and show how our framework can be applied to identify the underlying drivers of coexistence. Using an empirical case study, we demonstrate that the processes driving coexistence can be counterintuitive, and that our general predictive framework provides a better way to identify the true processes through which soil microbes affect coexistence.
\par



\section{Temporal dimension of species interaction}
Plant--soil feedbacks (PSF), the reciprocal interactions between plants and soil microbes, shape the structure of plant communities, but how the length of soil conditioning affects PSF strength is rarely studied. 
Using a chronosequence reconstructed from aerial photos, we characterized the soil microbial communities associated with four perennial dune plants and their turnover as plants aged. 
We also studied the resulting PSF in a greenhouse experiment that preserved age-specific soil properties.
For all plants, we found that their microbial communities changed with increasing conditioning time.
These compositional turnovers translated into a temporally varying PSF, and different plant--soil combinations showed different temporal development patterns.
With an general individual-based model, we further found that the temporal development patterns of PSF affected the convergence rate of plant community assembly. 
Our results suggest that future studies should take a dynamic perspective to understand the long-term consequences of PSF on plant community dynamics.
\par



\section{Author Contributions}
	Throughout my dissertation, I worked closely with many collaborators whose contributions are detailed below. These collaborators are listed as co-authors on the peer-reviewed manuscripts (or manuscripts in preparation) corresponding to each of my dissertation chapters.
\par


\noindent \textbf{Chapter 2: Linking modern coexistence theory and comtemporary niche theory}\\
\noindent Andrew Letten and Tadashi Fukami worked with me to develop the ideas for the research. I designed the simulation and performed the mathematical analysis. Andrew Letten wrote the first draft of the manuscript. Tadashi Fukami and I contributed to the writing and revision of the manuscript. Funding supporting this work came from the Center for Computational, Evolutionary, and Human Genomics (CEHG), the Department of Biology and the Terman Fellowship of Stanford University, and the National Science Foundation (NSF). The research has been published as a peer-reviewed manuscript in \textit{Ecological Monographs}.
\bigskip


\noindent \textbf{Chapter 3: Coexistence theory and the frequency-dependence of priority effects}\\
\noindent Andrew Letten worked with me to design the research. Andrew Letten assisted in numerical analyses and contributed to manuscript writing and revision. Funding supporting this work came from the Center for Computational, Evolutionary, and Human Genomics (CEHG), the Department of Biology of Stanford University, and the Studying Abroad Scholarship from the Ministry of Education, Taiwan. The research has been published as a peer-reviewed manuscript in \textit{Nature Ecology \& Evolution}.
\bigskip


\noindent \textbf{Chapter 4: Effects of soil microbes on plant competition: a perspective from modern coexistence theory}\\
\noindent Joe Wan worked with me to develop the research and formulate the analysis. He also worked with me on manuscript preparation and revision. Funding supporting this work came from the Department of Biology of Stanford University and the Studying Abroad Scholarship from the Ministry of Education, Taiwan. The research is under peer-review at \textit{Ecological Monographs}.
\bigskip


\noindent \textbf{Chapter 5: Testing chronosequence predictions with longitudinal data reveals microbial community convergence}\\
\noindent Tadashi Fukami worked with me to design the research. Tadashi Fukami and J. Nick Hendershot assisted in data analysis and manuscript preparation. Funding supporting this work came from the Department of Biology and the Terman Fellowship of Stanford University, the National Science Foundation (NSF), and the Studying Abroad Scholarship from the Ministry of Education, Taiwan.
\bigskip


\noindent \textbf{Chapter 6: Dynamic plant--soil microbe interactions: the neglected effect of soil conditioning time}\\
\noindent Tadashi Fukami worked with me to design the research and assisted in the analyzing the empirical data. Peter Zee and I developed the individual-based model and performed numerical simulations. Tadashi Fukami and Peter Zee contributed to manuscript preparation and revision. Funding supporting this work came from the Department of Biology and the Terman Fellowship of Stanford University, and the National Science Foundation (NSF), and the Studying Abroad Scholarship from the Ministry of Education, Taiwan. The research has been submitted for publication at \textit{Ecology Letters}.

